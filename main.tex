\documentclass{article}
\usepackage[utf8]{inputenc}


\title{Decal Syllabus}
\author{yutong_wu5, Esha Madhekar}
\date{September 2018}


\begin{document}

\maketitle


\section{Introduction to Dessert 137: Travel around Asia with your taste bud}



\subsection{Instructor Information}
Student Team: Amanda Ma, Esha Madhekar, Zihan Wen, Yutong Wu



\subsection{Course Description:}
Welcome to Dessert 137: Travel around Asia with your taste bud! This course is the first part of a series of course that special designed to introduce the beauty of Asian desert that will not only teach you history of the desserts with fun facts behind the scene, but also sharpen your cooking practical skills!

\subsection{What We Expect}
This class is a project-based, Pass/No-Pass 2 unit course with an emphasis on concepts that require no background but amount to significant pain points for those who are eager to learn more about Asian dessert. It will consist 6 projects that will be completed in class, several reading assignment and 6 1-page long reflections on the projects and reading assignments. The course is graded on satisfactory completion of these projects and reflections.
We encourage all interested students of all majors to enroll in this course, as long as you are interested in the material. There is no prerequisite for the course and all the student facilitators will provide you a broad support on whatever you need. Have fun and join the class if you are interested!


\subsection{Learning Objectives}
Throughout this course we encourage you to develop your understanding by being able to participate in labs every two weeks, and engage with the material during lecture. Throughout the course we hope to solidify your understanding and introduce you to the art of making Asian Desserts from three Asian countries: Thailand, India and China. We hope you can engage with the material and develop a strong understanding through the hands-on labs, reflections, and reading assignments!

\section{Resources}
\subsection{Required Text}
There are no required Text for the course but there will be several reading assignments will be given as specified in the course assignment section. 

\subsection{Recommended reading List}

Lucky Peach presents 101 easy Asian recipes / Peter Meehan 
There is a section in this book that focused on Asian dessert
The language of food : a linguist reads the menu / Dan Jurafsky
Although it does not specifically focus on Asian dessert, it is still a fun book to read that relates dessert and other food to linguistic.

Reading Assignment that assigned before class are recommended to be completed and will be mentioned in class.

Course Assignment

\section*{Theme 1: Indian Dessert}


\begin{table}[h]
\begin{tabular}{lllll}
 Lecture1 &  TBA & How to Make Jalebi & Esha Madhekar  \\
 Lab 1 &  TBA & Make Jalebi yourself! & Esha Madhekar  \\
 Lecture 2 &  TBA & How to Make Kaju Katli & Esha Madhekar  \\
 Lab 2 & TBA & Make Kaju Katli yourself! & Esha Madhekar 
\end{tabular}
\end{table}

 \section*{Theme 2: Chinese }\\
 \vspace{0.5em} \begin{tabular}{|l|l|l|l|}
\hline 
Lecture 1 & TBA & How to make a salted egg yolk ice cream mooncake & Zihan Wen\\
\hline
Lab 1 & TBA & Make a mooncake yourself!& Zihan Wen\\
\hline
Lecture 2 & TBA & How to make dragon beard candy & Zihan Wen\\
\hline 
Lab 2 & TBA & Make dragon beard candy yourself!& Zihan Wen\\
\hline
\end{tabular}


 \vspace{0.5em}
  \par\noindent\rule{\textwidth}{0.4pt}
  
\noindent\textbf{Readings and assignments}
\begin{itemize}
\item Watch a youtube video https://www.youtube.com/watch?v=fVmdWMLtEgY (Mooncake tutorial by Amanda tastes (7 min, optional)
\item Design your own mooncake flavor and list all the ingredients 
\item Watch a youtube video https://www.youtube.com/watch?v=Z2Zl64cSmpo("I Made The Impossible Cotton Candy From Ancient China 
\end{itemize}
  \par\noindent\rule{\textwidth}{0.4pt}
\textbf{Accommodations }
\begin{itemize}
\vspace{0.5em} 
\item\noindent The College will make reasonable accommodations for person with documented disabilities. Students should notify the Center for Disability Services office and their instructors of any special needs. Instructors should be notified as soon as possible.

 \end{itemize}
  \par\noindent\rule{\textwidth}{0.4pt}
 \section*{Course Schedule }\\
Course Policies

\section{Grading Rubrics} 


To pass this class you will need a passing grade of 60\%. This should be pretty doable as you will be graded on:

\begin{table}[h]
\begin{tabular}{lll}
 Attendance &  30\% \\
 Cooking/ Lab Work & 40\%  \\
 Reflections &  20\%  \\
 Participation & 10\% \\ 
\end{tabular}
\end{table}   

Below, each of the categories is explained in detail:
\subsection{Attendance}
 The attendance grade is based on how many classes you attend. Every class you will be given a cod that you have to input in a Google Form so we will know if you came to class! You have 1 unexcused absence after which  your attendance grade will decrease by a letter grade.

\subsection{Cooking/ Lab Work} This grade will be mostly based on the amount of effort and attentiveness you show during lab. Maintaining a clean workspace and leaving with a clean space will also be 50\% of this grade. Even though we hope that everyone will be able to leave with a well-made dish we know that his may not always be possible so this will only count towards 20\% of the cooking grade.

\subsection{Reflections} Your reflections will be graded for accuracy and including information contained in readings as well as discussed during class.

\subsection{Participation/Team Work} Teamwork during Cooking Labs as well and asking/answering questions during class will constitute most of this grade.


\end{document}



